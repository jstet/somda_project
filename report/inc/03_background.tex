\chapter{Background}
\label{sec:background}

The paper by Yasseri et al. was written in the wake of similar attempts to use digital traces to predict social outcomes. Distinctions of these attempts can be made by data sources and the nature of the predicted social phenomenon. A data source that is comparable to Wikipedia page views, because it originates from people looking for information, is web search data. For example, a study that the authors mention explores using web search data to predict consumer behavior across different domains \cite{Goel2010Oct}. While there are domain-specific differences in prediction strength, the paper's findings suggest that this generally produces significant results. Search data originating from Google searches has also been applied to predict doctor visits for influenza-like illness. This produced results with high accuracy \cite{Ginsberg2009Feb} for a limited time. A project by Google, called Google Flu Trends, attempted to provide this as a web service but famously failed in 2013. It has been hypothesized that the predictive power of Google search data is limited due to the nature of Google's search algorithm that biases results \cite{Lazer2014Mar}. While Wikipedia itself does not use algorithms that bias page views in the same way, their traffic could be affected indirectly by internet users being forwarded from search results. \par
Another data source used to predict social phenomena frequently both in academia and in commercial contexts is social media. In a recent literature review, Rousidis et al. distinguish phenomena predicted by social media data by assigning them to the fields of Finance, Marketing, and "Sociopolitical" \cite{Rousidis2020Mar}. The sociopolitical domain, which elections belong to as well, poses the most challenges for prediction; it has witnessed both successful prediction outcomes and notable failures. Another finding of Rousidis et al. is that among the data sources used as predictors, Twitter is the most popular (77\% of 43 studies). Yasseri et al. cite a study that claims to have found a statistically significant correlation between the presence of tweets mentioning candidates and their subsequent electoral performance \cite{DiGrazia2013Nov}. However, it has been argued by other authors that using Twitter data for prediction in the sociopolitical domain, especially for predicting elections is problematic or even fruitless. Suggested reasons include the existence of unanswered questions about the nature of political conversations on social media and representativity issues \cite{Gayo-Avello2011}. This has to be taken into account when using Wikipedia page views as a predictor. Although articles on Wikipedia are the results of a social process of multiple people contributing, Wikipedia page views should not be influenced by this in the case of the study to be replicated, as the relevant articles already exist, and it is not relevant how people interact with them. Representation issues however could influence the prediction in case Wikipedia pages are visited by a sample of people not representative of the electorate. A third reason for biased social media data has also been mentioned by  Yasseri et al. Following their proposed rational choice approach, they theorize that page views could be biased by voters seeking information on new political parties more frequently than on already established parties. A similar phenomenon has been implicitly described by a study \cite{Jungherr2011Apr} attempting to replicate an earlier study \cite{Tumasjan2010May} that claimed to be able to successfully predict the results of the German Election of 2009 using Twitter data. The original study only included the larger German parties, while the replication study also included the small "Pirate Party". This resulted in a predicted win for the Pirate Party, which did not happen in reality. The cause of the overprediction for this party may be the fact that this party was fairly new at that time or that it was controversial, causing it to be mentioned more than other parties without it having a proportional effect on election outcomes. \par
The frequent visits to Wikipedia pages during elections can be analyzed as a trend in the context of existing research on social media. Crane et al. introduced a model that can be utilized to classify the impact of an election on page views as a specific type of trend \cite{Crane2008Oct}. Applying this model, the social system is Wikipedia, and the external event is the election. Based on visualizations by Yasseri et al. (Figure 1), the peak of the trend, which corresponds to the days with the highest page views, accounts for more than 80\% of the total page views during the analyzed time period. According to Crane et al., this trend qualifies as Class 1 and is considered exogenous subcritical. While it is debatable whether Wikipedia can be considered a social system apart from the creation of articles, classifying the trend dynamics itself as exogenous subcritical is reasonable as there is less reason to research the specifics of the election once it has been held, resulting in a fast decay of page views to a significantly lower level compared to the peak.
\par
One motivation for the study by Yasseri et al. is a described lack of papers providing theories containing explanations for why digital traces have predictive power \cite{Lazer2009Feb}, which they are counteracting by providing a rational choice theory of information seeking. The authors explanation of this theory is limited, and they do not provide sources for what they are basing their theory on. Brief literature research reveals that rational choice theory has mainly been used to investigate the decision to vote itself, which led to a paradox known as the "voting paradox", describing the result that voting is irrational when the individual wants to optimize personal utility \cite{Bendor2003May, Martorana2011Nov}. One paper claims that "not much could be learned by taking a purely rational choice approach" when searching for explanations on why people seek political information during elections \cite{Ohr2001Dec}. Instead,  Ohr et al. propose a mixed approach that considers the social context in which voters are situated, as well as external and internal expectations that drive information-seeking behavior. They emphasize that information seeking serves as a rational tool to fulfill the aforementioned expections but also to enable the expressive component of voting and to engage with the entertaining aspects of elections. The proposed theory shares the same outcome as the theory by Yasseri et al., which is the decision to seek information. However, it presents an alternative and more comprehensive framework for understanding the predictive nature of Wikipedia page views in relation to election outcomes. \par
After Yasseri et al. , two other papers \cite{Salem2021Jun, Smith2017Sep} have used Wikipedia page views to predict elections. Both studies do not find that Wikipedia pageviews can independently predict electoral outcomes. Instead, they propose using Wikipedia pageviews as a complementary measure to enhance predictions made by other models, as it can explain variance that conventional information sources cannot. Yasseri et al.  set the groundwork for this by acknowledging that new parties and swing voters bias the page views.
  





