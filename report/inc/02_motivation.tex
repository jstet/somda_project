\chapter{Motivation}
\label{sec:motivation}

This project focuses on the use of Wikipedia traffic as a digital trace of internet users seeking information. Wikipedia traffic occurs when users access a Wikipedia page, resulting in a logged request to the server. This straightforward metric of tracking the number of views on specific Wikipedia pages over time can be leveraged for quantitative analysis to test different hypotheses related to information-seeking behavior and its implications. Implications can be of political nature when the subject of information-seeking is a social phenomenon like elections. In the paper titled "Wikipedia traffic data and electoral prediction: towards theoretically informed models" ~\cite{Yasseri2016Dec} the authors are investigating such implications. The objective of this project is to replicate the parts of findings of the paper while also extending the application of the methods explored in the paper to more recent data. \par 

Yasseri et al. are developing a theory on the relationship between views on election-related pages on Wikipedia and election outcomes. Following the rational choice paradigm, the theory is based on the assumption that voters are deciding for the party whose policies are maximizing their benefit. To do so, it is considered rational to seek information about the election and party policies. The authors cite studies indicating that a significant proportion of adults across different countries rely on Wikipedia as an information source ~\cite{Rainie2020May, Ofcom2015}. It can be assumed that this still holds. One piece of evidence is that Wikipedia is ranked as the 7th most visited website in the world in August 2023 ~\cite{BibEntry2023Aug}. The authors conclude that views on Wikipedia pages related to an election before the election itself are mainly due to voters seeking information and that one can therefore use page views as a predictor of election outcomes. 

Yasseri et al. bring forth two main hypotheses: 1. Page view numbers of general election pages are a predictor of election turnout and 2. Page view numbers of party pages are a predictor of party results. However, they theorize that there are factors that may moderate the prediction strength when it comes to party results, such as voters being more likely to seek information on new political parties and swing voters being more likely to seek information if they are considering changing their vote. Moreover, the authors argue that coverage of parties in the traditional media could weaken the prediction strength of Wikipedia page views. The paper's findings indicate that while Wikipedia may not provide detailed information on the exact results of votes, it does offer useful insights into changes in overall voter turnout and shifts in the percentage of votes obtained by specific parties. These results are based on outcomes of the European Parliament elections in 2009 and 2014 and related Wikipedia page view statistics and traditional news coverage from the time of the elections. 

In this project, I am extending the data to the 2019 European Parliament elections, while also gathering page view statistics on the previous two elections, because the authors did not include this data with their paper. My main objective is to replicate testing one the author's first hypotheses about Wikipedia page views being a predictor for election turnout.
