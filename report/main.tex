%%%%%%%%%%%%%%%%%%%%%%%%%%%%%%%%%%%%%%%%%%%%%%%%%%%%%%%%%
%                                                       %
% Vorlage fuer die Erstellung von PDF-Dokumenten        %
% im Corporate Design der Universität Konstanz          %
% Stand: 14.09.2016 - Michael Brendle (Version 0.4)     %
%                                                       %
%%%%%%%%%%%%%%%%%%%%%%%%%%%%%%%%%%%%%%%%%%%%%%%%%%%%%%%%%

%%%%%%%%%%%%%%%%%%%%%%%%%%%%%%%%%%%%%%%%%%%%%%%%%%%%%%%%%
% Dokumentenklasse scrreprt                             %
%%%%%%%%%%%%%%%%%%%%%%%%%%%%%%%%%%%%%%%%%%%%%%%%%%%%%%%%%

\documentclass[12pt, rgb, english, parskip=full, oneside, ]{scrreprt}

\usepackage{graphicx}
\usepackage{multicol}
\usepackage{etoolbox}
\makeatletter
\patchcmd{\chapter}{\if@openright\cleardoublepage\else\clearpage\fi}{}{}{}
\makeatother

\usepackage{setspace}
\setstretch{1.05}

%%%%%%%%%%%%%%%%%%%%%%%%%%%%%%%%
% Paket Theme Konstanz         %
%%%%%%%%%%%%%%%%%%%%%%%%%%%%%%%%

%\usepackage{themeKonstanzXelatexAddOn} % XeLaTeX mit Schriftart Arial, VOR dem Standardpaket importieren
\usepackage{themeKonstanz} % Muss immer verwendet werden (Standardpaket)
\usepackage{themeKonstanzStyleAddOn} % Style Add-On für andere Überschriften, NACH dem Standardpaket importieren



% preventing widow/orphan lines
\widowpenalty10000
\clubpenalty10000

% underline with dotted line
\newcommand{\uloosdash}[1]{%
    \tikz[baseline=(todotted.base)]{
        \node[inner sep=1.5pt,outer sep=0pt] (todotted) {#1};
        \draw[loosely dashed] (todotted.south west) -- (todotted.south east);
    }%
}%

% underline with dashed line
\newcommand{\uldotted}[1]{%
    \tikz[baseline=(todotted.base)]{
        \node[inner sep=1.5pt,outer sep=0pt] (todotted) {#1};
        \draw[dotted] (todotted.south west) -- (todotted.south east);
    }%
}%

% create circle box with text
% #1 size -- #2 draw color -- #3 fill color -- #4 text color -- #5 text
\newcommand{\circlebox}[5]{\resizebox{#1}{!}{\tikz \node[draw=#2, fill=#3, circle, scale=1, text=#4, inner sep=0.5mm, font=\bfseries] at (0,0) {#5};}}

% create rectangle box with text
% #1 size -- #2 draw color -- #3 fill color -- #4 text color -- #5 text
\newcommand{\rectanglebox}[5]{\resizebox{#1}{!}{\tikz \node[draw=#2, fill=#3, rectangle, scale=1, text=#4, inner sep=1mm, font=\bfseries] at (0,0) {#5};}}

\newcommand{\dquote}[1]{``#1''}
\newcommand{\squote}[1]{`#1'}
\newcommand{\etc}{etc}
\newcommand{\ie}{i.e.}
\newcommand{\eg}{e.g.}
\newcommand{\cf}{cf.}
\newcommand{\etal}{et al.}

% Additional colors
\definecolor{mdgreen}{RGB}{0, 204, 0}
\definecolor{ltblue}{RGB}{153, 204, 255}
\definecolor{dkblue}{RGB}{0, 143, 204}
\definecolor{dkgrey}{RGB}{128, 128, 128}
\definecolor{ltgrey}{RGB}{160, 160, 160}

\definecolor{yellow}{RGB}{255, 255, 0}
\definecolor{red}{RGB}{255, 50, 50}

\newcommand{\todo}[1]{\noindent \textcolor{red}{TODO: #1}}

%%%%%%%%%%%%%%%%%%%%%%%%%%%%%%%%
% Dokumentinformationen        %
%%%%%%%%%%%%%%%%%%%%%%%%%%%%%%%%

\date{20.08.2023}
\year{2023}
\author{Jonas Stettner (01/1152625)}
\title{Replication Study: Wikipedia traffic data and electoral prediction}
\subtitle{}
\unisection{Social Media Data Analysis }
\department{Department of Politics and Public Administration}
\supervisorOne{Prof. David Garcia}



\headFoot{14}

%%%% Default for rm family
%\renewcommand\familydefault{\rmdefault}
% \setmathfont{OT1}

%%%%%%%%%%%%%%%%%%%%%%%%%%%%%%%%%%%%%%%%%%%%%%%%%%%%%%%%%
% Begin vom Dokument                                    %
%%%%%%%%%%%%%%%%%%%%%%%%%%%%%%%%%%%%%%%%%%%%%%%%%%%%%%%%%

\begin{document}




%%%%%%%%%%%%%%%%%%%%%%%%%%%%%%%%
% Thesis-Titelseite            %
%%%%%%%%%%%%%%%%%%%%%%%%%%%%%%%%

\thesistitlepage[language=english]{Project Report}

\rmfamily % Auskommentieren für eine Schrift mit serifen / Kommentieren für eine serifenlose Schrift
\normalsize

\newpage




%%%%%%%%%%%%%%%%%%%%%%%%%%%%%%%%
% Inhaltsverzeichnis           %
%%%%%%%%%%%%%%%%%%%%%%%%%%%%%%%%

% Das Inhaltsverzeichnis kann wie gewohnt mit dem Makro
%
%    \tableofcontents
%
% erstellt werden.

\tableofcontents

\newpage


%%%%%%%%%%%%%%%%%%%%%%%%%%%%%%%%
% Abbildungsverzeichnis        %
%%%%%%%%%%%%%%%%%%%%%%%%%%%%%%%%

% \listoffigures
% \addcontentsline{toc}{chapter}{Abbildungsverzeichnis}



%%%%%%%%%%%%%%%%%%%%%%%%%%%%%%%%
% Tabellenverzeichnis          %
%%%%%%%%%%%%%%%%%%%%%%%%%%%%%%%%

% \listoftables
% \addcontentsline{toc}{chapter}{Tabellenverzeichnis}



%%%%%%%%%%%%%%%%%%%%%%%%%%%%%%%%
% Kapitel, Abschnitte, ...     %
%%%%%%%%%%%%%%%%%%%%%%%%%%%%%%%%

\pagenumbering{arabic}

\chapter{Motivation}
\label{sec:motivation}

This project focuses on the use of Wikipedia traffic as a digital trace of internet users seeking information. Wikipedia traffic occurs when users access a Wikipedia page, resulting in a logged request to the server. This straightforward metric of tracking the number of views on specific Wikipedia pages over time can be leveraged for quantitative analysis to test different hypotheses related to information-seeking behavior and its implications. Implications can be of political nature when the subject of information-seeking is a social phenomenon like elections. In the paper titled "Wikipedia traffic data and electoral prediction: towards theoretically informed models" ~\cite{Yasseri2016Dec} the authors are investigating such implications. The objective of this project is to replicate the parts of findings of the paper while also extending the application of the methods explored in the paper to more recent data. \par 

Yasseri et al. are developing a theory on the relationship between views on election-related pages on Wikipedia and election outcomes. Following the rational choice paradigm, the theory is based on the assumption that voters are deciding for the party whose policies are maximizing their benefit. To do so, it is considered rational to seek information about the election and party policies. The authors cite studies indicating that a significant proportion of adults across different countries rely on Wikipedia as an information source ~\cite{Rainie2020May, Ofcom2015}. It can be assumed that this still holds. One piece of evidence is that Wikipedia is ranked as the 7th most visited website in the world in August 2023 ~\cite{BibEntry2023Aug}. The authors conclude that views on Wikipedia pages related to an election before the election itself are mainly due to voters seeking information and that one can therefore use page views as a predictor of election outcomes. Yasseri et al. bring forth two main hypotheses: 1. Page view numbers of general election pages are a predictor of election turnout and 2. Page view numbers of party pages are a predictor of party results. However, they theorize that there are factors that may moderate the prediction strength when it comes to party results, such as voters being more likely to seek information on new political parties and swing voters being more likely to seek information if they are considering changing their vote. Moreover, the authors argue that coverage of parties in the traditional media could weaken the prediction strength of Wikipedia page views. The paper's findings indicate that while Wikipedia may not provide detailed information on the exact results of votes, it does offer useful insights into changes in overall voter turnout and shifts in the percentage of votes obtained by specific parties. These results are based on outcomes of the European Parliament elections in 2009 and 2014 and related Wikipedia page view statistics and traditional news coverage from the time of the elections. In this project, I am extending the data to the 2019 European Parliament elections, while also gathering page view statistics on the previous two elections, because the authors did not include this data with their paper. My main objective is to replicate testing one the author's first hypotheses about Wikipedia page views being a predictor for election turnout.

\chapter{Background}
\label{sec:background}

The paper by Yasseri et al. was written in the wake of similar attempts to use digital traces to predict social outcomes. Distinctions of these attempts can be made by data sources and the nature of the predicted social phenomenon. A data source that is comparable to Wikipedia page views, because it originates from people looking for information, is web search data. For example, a study that the authors mention explores using web search data to predict consumer behavior across different domains \cite{Goel2010Oct}. While there are domain-specific differences in prediction strength, the paper's findings suggest that this generally produces significant results. Search data originating from Google searches has also been applied to predict doctor visits for influenza-like illness. This produced results with high accuracy \cite{Ginsberg2009Feb} for a limited time. A project by Google, called Google Flu Trends, attempted to provide this as a web service but famously failed in 2013. It has been hypothesized that the predictive power of Google search data is limited due to the nature of Google's search algorithm that biases results \cite{Lazer2014Mar}. While Wikipedia itself does not use algorithms that bias page views in the same way, their traffic could be affected indirectly by internet users being forwarded from search results. \par
Another data source used to predict social phenomena frequently both in academia and in commercial contexts is social media. In a recent literature review, Rousidis et al. distinguish phenomena predicted by social media data by assigning them to the fields of Finance, Marketing, and "Sociopolitical" \cite{Rousidis2020Mar}. The sociopolitical domain, which elections belong to as well, poses the most challenges for prediction; it has witnessed both successful prediction outcomes and notable failures. Another finding of Rousidis et al. is that among the data sources used as predictors, Twitter is the most popular (77\% of 43 studies). Yasseri et al. cite a study that claims to have found a statistically significant correlation between the presence of tweets mentioning candidates and their subsequent electoral performance \cite{DiGrazia2013Nov}. However, it has been argued by other authors that using Twitter data for prediction in the sociopolitical domain, especially for predicting elections is problematic or even fruitless. Suggested reasons include the existence of unanswered questions about the nature of political conversations on social media and representativity issues \cite{Gayo-Avello2011}. This has to be taken into account when using Wikipedia page views as a predictor. Although articles on Wikipedia are the results of a social process of multiple people contributing, Wikipedia page views should not be influenced by this in the case of the study to be replicated, as the relevant articles already exist, and it is not relevant how people interact with them. Representation issues however could influence the prediction in case Wikipedia pages are visited by a sample of people not representative of the electorate. A third reason for biased social media data has also been mentioned by  Yasseri et al. Following their proposed rational choice approach, they theorize that page views could be biased by voters seeking information on new political parties more frequently than on already established parties. A similar phenomenon has been implicitly described by a study \cite{Jungherr2011Apr} attempting to replicate an earlier study \cite{Tumasjan2010May} that claimed to be able to successfully predict the results of the German Election of 2009 using Twitter data. The original study only included the larger German parties, while the replication study also included the small "Pirate Party". This resulted in a predicted win for the Pirate Party, which did not happen in reality. The cause of the overprediction for this party may be the fact that this party was fairly new at that time or that it was controversial, causing it to be mentioned more than other parties without it having a proportional effect on election outcomes. \par
The frequent visits to Wikipedia pages during elections can be analyzed as a trend in the context of existing research on social media. Crane et al. introduced a model that can be utilized to classify the impact of an election on page views as a specific type of trend \cite{Crane2008Oct}. Applying this model, the social system is Wikipedia, and the external event is the election. Based on visualizations by Yasseri et al. (Figure 1), the peak of the trend, which corresponds to the days with the highest page views, accounts for more than 80\% of the total page views during the analyzed time period. According to Crane et al., this trend qualifies as Class 1 and is considered exogenous subcritical. While it is debatable whether Wikipedia can be considered a social system apart from the creation of articles, classifying the trend dynamics itself as exogenous subcritical is reasonable as there is less reason to research the specifics of the election once it has been held, resulting in a fast decay of page views to a significantly lower level compared to the peak.
\par
One motivation for the study by Yasseri et al. is a described lack of papers providing theories containing explanations for why digital traces have predictive power \cite{Lazer2009Feb}, which they are counteracting by providing a rational choice theory of information seeking. The authors explanation of this theory is limited, and they do not provide sources for what they are basing their theory on. Brief literature research reveals that rational choice theory has mainly been used to investigate the decision to vote itself, which led to a paradox known as the "voting paradox", describing the result that voting is irrational when the individual wants to optimize personal utility \cite{Bendor2003May, Martorana2011Nov}. One paper claims that "not much could be learned by taking a purely rational choice approach" when searching for explanations on why people seek political information during elections \cite{Ohr2001Dec}. Instead,  Ohr et al. propose a mixed approach that considers the social context in which voters are situated, as well as external and internal expectations that drive information-seeking behavior. They emphasize that information seeking serves as a rational tool to fulfill the aforementioned expections but also to enable the expressive component of voting and to engage with the entertaining aspects of elections. The proposed theory shares the same outcome as the theory by Yasseri et al., which is the decision to seek information. However, it presents an alternative and more comprehensive framework for understanding the predictive nature of Wikipedia page views in relation to election outcomes. \par
After Yasseri et al., two other papers \cite{Salem2021Jun, Smith2017Sep} have used Wikipedia page views to predict elections. Both studies do not find that Wikipedia pageviews can independently predict electoral outcomes. Instead, they propose using Wikipedia pageviews as a complementary measure to enhance predictions made by other models, as it can explain variance that conventional information sources cannot. Yasseri et al.  set the groundwork for this by acknowledging that new parties and swing voters bias the page views.
  






\chapter{Data}
\label{sec:data}
This project aims to analyze the relationship between Wikipedia page views and election outcomes in the EU elections of 2009, 2014, and 2019. To achieve this, two key metrics are required: Wikipedia page views for the relevant Wikipedia pages on the elections in the included countries, and the election turnout data. To obtain these metrics, the project retrieves data from two sources: Wikipedia Page View Dumps and the official EU website containing election results. The Wikipedia page views are accessed through downloadable files, as the API Wikimedia provides for this purpose only provides data from 2015 onwards. These files contain page views from all Wikimedia projects in all languages and can be quite large. In 2019, for example, they range from 400-500 MB uncompressed to around 4 GB uncompressed. Another challenge is that the data before December 2011 has a different format both in content and on the time basis the files exist at. This necessitates separate data processing steps to handle these differences.

To process the page view dump files, the project uses the Python framework and cloud computing provider modal.com. Intermediate results are stored on a remote S3 server.  Files corresponding to 14 days before and after the election date are downloaded and processed concurrently, handling 100 files at a time. The time span of 28 days is chosen to limit the amount of data to be processed as it would have been too resource and time-consuming for the scope of this project otherwise. The files are then decompressed in chunks and filtered line by line to include only statistics for articles within the Wikipedia project. Each filtered chunk is then written to a parquet file. To query the election article page views, a manually compiled dictionary containing the names of the election articles in the included at the election dates is used. One SQL query per country per file is made using an in-memory duckDB database using python code in the following manner:

\begin{verbatim}
    temp = duckdb.query(f"""
    SELECT *
    FROM '{input_filepath}'
    WHERE  contains(article_title, '{page_name}') AND wikicode = '{wikicode}'
\end{verbatim}

The results are stored in individual Pandas dataframes, where each row represents the hourly page views for the Wikipedia election page in a specific country. Since all the steps are performed concurrently, the final step involves concatenating these dataframes.





\chapter{Methods}
\label{sec:methods}

In this replication study, I utilize methods that overlap with those employed in the paper by Yasseri et al. The objective is to analyze the relationship between Wikipedia page views and turnout in the EU elections by employing the Pearson Correlation Coefficient. In the analysis, I consider the relative change in page views as the independent variable and the relative change in turnout between two elections as the dependent variable. While Yasseri et al. do not provide explicit details on how they calculated the "volume of attention in the build-up phase" used to determine the relative change in page views, I assume it refers to the summation of all page views within a predetermined timeframe preceding the election date. The formula for relativ change is obviously:$$
\text{{Relative Change}} = \frac{{\text{{new value}} - \text{{initial value}}}}{{\text{{initial value}}}}
$$
Pearson's correlation coefficient is calculated by dividing the covariance of the two variables by the product of their standard deviations, but in this project this is automated using the \texttt{scipy.stats.pearsonr} function. The Pearson Correlation Coefficient is a commonly used statistical measure that helps assess the strength and direction of relationships between variables. It is valuable for understanding the level of association between two variables and can be utilized to uncover patterns or trends within the data. The strength of the relationship, indicated by how close the value is to 1 (or -1), can validate or refute the hypothesis that Wikipedia page views are a predictor for turnout.
\chapter{Results}
\label{sec:results}

Comparing the lineplot of page views from 2009 as shown in the paper by Yasseri et al. (Figure 1) to the lineplot recreated in Figure \ref{fig:figure1} of this report, it is evident that the two lineplots exhibit close similarity. The peak in page views aligns with the date of the 2009 European Parliament Elections, indicating a clear correlation between the election and the increased interest in related Wikipedia articles. 

\begin{figure*}[t!]
    \centering
    \includegraphics[width=\textwidth]{fig/lineplot_2009.png}
    \caption{'Normalized Wikipedia Election Page Views two weeks before and after the 2009 European Parliament Elections'}
    \label{fig:figure1}
\end{figure*} 
However, when visualizing the data from 2014 and 2019 in the same way, it becomes apparent that the patterns differ from those observed in the 2009 data. The peaks and fluctuations in page views do not align as closely with the election dates in these years. 
\begin{figure*}[t!]
    \centering
    \includegraphics[width=\textwidth]{fig/lineplot_2014_2019.png}
    \caption{'Normalized Wikipedia Election Page Views two weeks before and after the 2014 (left) and 2019 (right) European Parliament Elections'}
    \label{fig:figure2}
\end{figure*}
Taking a closer look at absolute values of France, a country  where the peak does not correspond to the election date, it becomes apparent that there must be errors in the data or its retrieval, as page views are at 0 or below 10 before a seemingly random peak at around the 5th of June. Further analysis has to be performed under the assumption that the data is erroneous.

\chapter{Discussion}
\label{sec:discussion}

Evaluate answers to the question and their reliability. Assess limitations and critique. Relate to results from other work or replicated paper


%%%%%%%%%%%%%%%%%%%%%%%%%%%%%%%%
% Literaturverzeichnis         %
%%%%%%%%%%%%%%%%%%%%%%%%%%%%%%%%

% Zum Schluss kann noch das Literaturvzerzeichnis hinzugefügt
% werden.
%
% Damit es ebenfalls im Inhaltsverzeichnis gefunden werden kann,
% sollte das Literaturverzeichnis mit dem Makro
%
%    \addcontentsline{toc}{chapter}{Literaturverzeichnis}
%
% hinzugefügt werden.



\addcontentsline{toc}{chapter}{Bibliography}
\bibliographystyle{apalike}
\bibliography{main.bib}


%%%%%%%%%%%%%%%%%%%%%%%%%%%%%%%%%%%%%%%%%%%%%%%%%%%%%%%%%
% Ende vom Dokument                                     %
%%%%%%%%%%%%%%%%%%%%%%%%%%%%%%%%%%%%%%%%%%%%%%%%%%%%%%%%%

\end{document}
